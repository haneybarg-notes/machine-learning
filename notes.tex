% Created 2019-05-28 Tue 18:36
% Intended LaTeX compiler: pdflatex
\documentclass[11pt]{article}
\usepackage[utf8]{inputenc}
\usepackage[T1]{fontenc}
\usepackage{graphicx}
\usepackage{grffile}
\usepackage{longtable}
\usepackage{wrapfig}
\usepackage{rotating}
\usepackage[normalem]{ulem}
\usepackage{amsmath}
\usepackage{textcomp}
\usepackage{amssymb}
\usepackage{capt-of}
\usepackage{hyperref}
\usepackage[margin=2cm]{geometry}
\usepackage{enumitem}
\DeclareMathOperator{\sign}{sign}
\setlength{\parindent}{0cm}
\usepackage{pgfplots}
\pgfplotsset{compat=1.11}
\usetikzlibrary{arrows, decorations.markings}
\usetikzlibrary{3d}
\usetikzlibrary{shapes.geometric,decorations.fractals,shadows}
\date{\today}
\title{}
\hypersetup{
 pdfauthor={},
 pdftitle={},
 pdfkeywords={},
 pdfsubject={},
 pdfcreator={Emacs 26.2 (Org mode 9.2)},
 pdflang={English}}
\begin{document}

\tableofcontents


\section{Deep Learning}
\label{sec:org2ee12e0}
Utilized with unestructred data (Ex: image, text, time series).

We will study:
\begin{itemize}
\item Computer Vision
\item NLP (natural language processing)
\end{itemize}

\subsection{Computer Vision}
\label{sec:org0ceb043}
Chalenges: high dimensions and different inputs for the same object when
distinguishing images.

State of the art classifies better than humans.

CNN = convolutional neural networks (to learn these boxes, it needs a math operation
called convolution or correlation).

In the top layers, there are more smaller boxes. In the last layer, the is a vector
representation of all the patterns.

Lower layers catch very simple patterns, second layer starts getting forms. With more
layers, it gets more complex objects.

\subsubsection{Convolution}
\label{sec:org68a90c4}
Returns a number. Capacity of checking if the boxes are good (extract features from
input). Edges and corners are disavantaged in this.

\subsubsection{Pooling}
\label{sec:org0948903}
\begin{itemize}
\item \textbf{Max:} 99\% of times.
\item Min
\item Average
\end{itemize}

Reduce the dimentionaly of the boxes. Superior limit. Corresponds spatially to the
orginal image.
\end{document}
